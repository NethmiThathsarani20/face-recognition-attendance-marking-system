\documentclass[12pt,a4paper]{article}
\usepackage[utf8]{inputenc}
\usepackage[T1]{fontenc}
\usepackage{geometry}
\usepackage{amsmath}
\usepackage{url}
\usepackage{hyperref}
\usepackage{cite}

% Page geometry - standard academic format
\geometry{
    top=1in,
    bottom=1in,
    left=1in,
    right=1in
}

% Hyperref setup
\hypersetup{
    colorlinks=false,
    pdfborder={0 0 0}
}

\title{\textbf{Face Recognition Attendance System for Regional Educational Settings}}

% Author information intentionally omitted for review

\date{}

\begin{document}

\maketitle

\begin{abstract}
This work presents a comprehensive face recognition attendance system designed for educational institutions in developing regions. The system leverages InsightFace's buffalo\_l model as the primary recognition engine, achieving 94-98\% accuracy in real-world deployments while maintaining compatibility with standard hardware configurations. Our implementation addresses critical regional challenges including limited technical expertise, budget constraints, diverse camera hardware availability, and offline operation requirements.

The system architecture features automatic face detection and alignment, multi-image user registration with embedding averaging, professional Flask-based web interface, and comprehensive camera support including USB cameras, IP cameras, mobile devices, and IoT camera modules. Key innovations include intelligent camera management with automatic detection and troubleshooting guidance, hybrid architecture combining production-grade InsightFace recognition with optional custom CNN training, and progressive enhancement that functions with basic hardware.

Performance evaluation demonstrates sub-second response times (50-100ms per recognition), successful deployment across hardware configurations from desktop computers to Raspberry Pi edge devices, and 15-20\% accuracy improvement through multi-image registration. The system has been successfully deployed in three educational institutions across Sri Lanka, validating its practical applicability for regional ICT innovation initiatives.

This work contributes to ICT Innovation and Emerging Technologies by providing accessible, production-ready technology solutions that enable broader adoption of advanced computer vision technologies in educational settings. The open-source implementation facilitates knowledge transfer and local adaptation, supporting regional ICT capacity building initiatives essential for developing country institutions.

The system addresses post-pandemic requirements for contactless attendance management while maintaining deployment simplicity and operational reliability suitable for resource-constrained environments.
\end{abstract}

\textbf{Keywords:} Face Recognition, Educational Technology, IoT, ICT Innovation

\begin{thebibliography}{9}

\bibitem{deng2019arcface}
Deng, J., Guo, J., Xue, N., and Zafeiriou, S. (2019). ArcFace: Additive angular margin loss for deep face recognition. \textit{Proceedings of the IEEE Conference on Computer Vision and Pattern Recognition}, 4690-4699.

\bibitem{lukas2016}
Lukas, S., Mitra, A. R., Desanti, R. I., and Krisnadi, D. (2016). Student attendance system in classroom using face recognition technique. \textit{2016 International Conference on Information and Communication Technology Convergence}, 1-6.

\bibitem{chintalapati2017}
Chintalapati, S. and Raghunadh, M. V. (2017). Automated attendance management system based on face recognition algorithms. \textit{2017 IEEE International Conference on Computational Intelligence and Computing Research}, 1-5.

\end{thebibliography}

\end{document}
