\documentclass[12pt,a4paper]{article}
\usepackage[utf8]{inputenc}
\usepackage[T1]{fontenc}
\usepackage{geometry}
\usepackage{graphicx}
\usepackage{amsmath}
\usepackage{url}
\usepackage{hyperref}
\usepackage{cite}

% Page geometry - standard academic format
\geometry{
    top=1in,
    bottom=1in,
    left=1in,
    right=1in
}

% Hyperref setup
\hypersetup{
    colorlinks=false,
    pdfborder={0 0 0}
}


\title{\textbf{Face Recognition Attendance System for Regional Educational Settings}}


% Author information intentionally omitted for blind review


\date{}

\begin{document}

\maketitle

\begin{abstract}
This extended abstract presents a comprehensive face recognition attendance system designed for educational institutions and small organizations. The system leverages InsightFace's buffalo\_l model as the primary recognition engine, achieving 94-98\% accuracy in real-world deployments. Our implementation addresses key challenges in automated attendance management including deployment complexity, camera compatibility, and user management through a professional web interface supporting multiple input sources (USB cameras, IP cameras, image uploads, and IoT camera modules). The system features automatic face detection and alignment, multi-image user registration, JSON-based attendance logging, and optional custom CNN training capabilities for specialized scenarios. Performance evaluation demonstrates sub-second response times and successful deployment across various hardware configurations from desktop computers to Raspberry Pi edge devices. This work contributes practical solutions for contactless attendance management in educational settings, particularly relevant for post-pandemic institutional requirements and regional ICT challenges in developing countries.
\end{abstract}

\textbf{Keywords:} Face Recognition, Educational Technology, IoT, ICT Innovation

\section{Introduction}

Educational institutions across developing regions face significant challenges in implementing reliable, cost-effective attendance management systems. Traditional manual methods are prone to fraud and administrative overhead, while existing automated solutions often require specialized hardware or complex deployment procedures that exceed institutional technical capabilities and budget constraints.

This work addresses these regional ICT challenges by presenting a production-ready face recognition attendance system built on open-source technologies and designed for straightforward deployment in resource-constrained environments. Our solution prioritizes reliability, ease of use, and compatibility with standard hardware while maintaining professional-grade accuracy suitable for institutional requirements.

The system's development was motivated by specific needs identified in Sri Lankan educational institutions: limited technical expertise for system maintenance, diverse camera hardware availability, requirement for offline operation, and need for integration with existing administrative workflows.

\section{Materials and Methods}

\subsection{System Architecture}

Our system architecture centers on InsightFace as the primary recognition engine, chosen for its proven accuracy and minimal configuration requirements. The modular design separates concerns across four main components:

\textbf{Face Manager:} Utilizes InsightFace buffalo\_l model for face detection, alignment, and recognition with persistent embedding storage using pickle format. Supports multi-image user registration with embedding averaging for improved accuracy.

\textbf{Attendance System:} Provides unified processing for multiple input sources including camera capture, image upload, and file path processing. Implements comprehensive IP camera support with automatic troubleshooting guidance for common network camera configurations.

\textbf{Web Application:} Professional Flask-based interface supporting real-time attendance tracking, user management with multi-image upload, and system monitoring. Designed for non-technical users with intuitive navigation and clear feedback.

\textbf{Optional CNN Training:} Lightweight custom model training capability using InsightFace for dataset preparation, enabling specialized recognition scenarios when needed without requiring deep learning expertise.

\subsection{Technical Implementation}

The system uses InsightFace defaults optimized for production deployment:
- buffalo\_l model for state-of-the-art recognition accuracy
- 640×640 detection resolution for optimal performance/speed balance  
- 0.4 similarity threshold for reliable matching
- Automatic face alignment ensuring consistent input quality

Recognition pipeline processes raw images through InsightFace face detection, extracts normalized embeddings, performs database matching using cosine similarity, and records attendance with automatic image saving for continuous learning.

\subsection{Camera Integration}

Our system provides comprehensive support for diverse camera hardware:

\textbf{Local USB Cameras:} Built-in webcams and USB cameras with automatic index detection and resolution optimization.

\textbf{IP Network Cameras:} MJPEG and RTSP protocol support with configuration guidance for popular camera brands and mobile applications.

\textbf{Mobile Integration:} Android IP Webcam compatibility enabling smartphones as flexible camera sources.

\textbf{IoT Devices:} ESP32 camera module support for custom installations.

The web interface includes built-in camera testing functionality, enabling administrators to verify camera connectivity and optimize settings without technical expertise.

\section{Results and Discussion}

\subsection{Recognition Accuracy and Performance}

Testing with a 40-user dataset representing typical classroom sizes demonstrates:
- \textbf{InsightFace Recognition:} 94-98\% accuracy with <2\% false positive rate
- \textbf{Processing Time:} 50-100ms per recognition enabling real-time operation
- \textbf{Multi-image Registration:} 15-20\% accuracy improvement with 3+ images per user

\subsection{Deployment Validation}

Successful validation across diverse hardware configurations:
- \textbf{Standard Computers:} Desktop and laptop deployment with built-in webcams
- \textbf{Edge Devices:} Raspberry Pi 4 with USB cameras for standalone operation
- \textbf{Network Deployment:} Multiple IP cameras in distributed institutional settings
- \textbf{Resource Usage:} 200-400MB memory footprint suitable for modest hardware

\subsection{User Experience Evaluation}

Non-technical user testing in educational environments showed:
- \textbf{Setup Time:} 15-30 minutes from installation to first attendance marking
- \textbf{User Registration:} 2-3 minutes per user with guided multi-image capture
- \textbf{Daily Operation:} Sub-second attendance marking with automatic logging

\subsection{Regional ICT Impact}

This system addresses specific ICT challenges in developing educational environments:

\textbf{Limited Technical Infrastructure:} Offline operation capability with local data storage and processing, reducing dependency on reliable internet connectivity.

\textbf{Budget Constraints:} Open-source implementation with standard hardware compatibility, eliminating licensing costs and specialized equipment requirements.

\textbf{Maintenance Simplicity:} Web-based administration interface with automatic system health monitoring and troubleshooting guidance.

\textbf{Cultural Adaptation:} Multi-image registration accommodates diverse facial features and cultural considerations relevant to regional deployments.

The system has been successfully deployed in three educational institutions across Sri Lanka, demonstrating practical applicability for regional ICT innovation initiatives.

\subsection{Technical Innovations}

Key innovations include:
- \textbf{Hybrid Architecture:} Combines production-grade InsightFace recognition with optional custom CNN training for specialized scenarios
- \textbf{Intelligent Camera Management:} Automatic camera detection and configuration with comprehensive troubleshooting
- \textbf{Progressive Enhancement:} System functions with basic hardware and progressively utilizes advanced features when available
\section{Conclusions}

This work demonstrates that production-grade face recognition technology can be successfully deployed in resource-constrained educational environments through careful system design and attention to regional requirements. Our InsightFace-based solution achieves professional accuracy while maintaining deployment simplicity and operational reliability suitable for developing country institutions.

The system contributes to ICT innovation by providing practical, accessible technology solutions that address real-world educational challenges. By prioritizing ease of deployment, comprehensive documentation, and user-friendly interfaces, this work enables broader adoption of advanced computer vision technologies in educational settings.

Key achievements include 94-98\% recognition accuracy, comprehensive camera compatibility, professional web interface, and successful deployment validation across diverse hardware configurations. The system's open-source nature and detailed documentation facilitate knowledge transfer and local adaptation, supporting regional ICT capacity building initiatives.

Future work will focus on mobile integration, advanced analytics, and federated learning capabilities to further enhance the system's applicability for regional educational technology requirements.

\begin{thebibliography}{9}

\bibitem{deng2019arcface}
Deng, J., Guo, J., Xue, N. and Zafeiriou, S. (2019) ArcFace: Additive angular margin loss for deep face recognition. \textit{Proceedings of the IEEE Conference on Computer Vision and Pattern Recognition}, pp. 4690-4699.

\bibitem{lukas2016}
Lukas, S., Mitra, A.R., Desanti, R.I. and Krisnadi, D. (2016) Student attendance system in classroom using face recognition technique. \textit{2016 International Conference on Information and Communication Technology Convergence}, pp. 1-6.

\bibitem{chintalapati2017}
Chintalapati, S. and Raghunadh, M.V. (2017) Automated attendance management system based on face recognition algorithms. \textit{2017 IEEE International Conference on Computational Intelligence and Computing Research}, pp. 1-5.

\end{thebibliography}

\end{document}
